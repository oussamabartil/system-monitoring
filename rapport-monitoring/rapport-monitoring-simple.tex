\documentclass[12pt,a4paper]{report}
\usepackage[utf8]{inputenc}
\usepackage[french]{babel}
\usepackage[T1]{fontenc}
\usepackage{geometry}
\usepackage{graphicx}
\usepackage{float}
\usepackage{fancyhdr}
\usepackage{titlesec}
\usepackage{tocloft}
\usepackage{xcolor}
\usepackage{hyperref}
\usepackage{tikz}
\usepackage{booktabs}
\usepackage{longtable}
\usepackage{array}

% Configuration de la page
\geometry{left=3cm,right=2cm,top=2.5cm,bottom=2.5cm}

% Configuration des couleurs
\definecolor{primaryblue}{RGB}{0,102,204}
\definecolor{secondaryblue}{RGB}{51,153,255}
\definecolor{lightgray}{RGB}{245,245,245}
\definecolor{darkgray}{RGB}{64,64,64}

% Configuration des en-têtes et pieds de page
\pagestyle{fancy}
\fancyhf{}
\fancyhead[L]{\textcolor{primaryblue}{\textbf{Rapport de Monitoring}}}
\fancyhead[R]{\textcolor{primaryblue}{\thepage}}
\fancyfoot[C]{\textcolor{darkgray}{\small Système de Monitoring Prometheus \& Grafana}}

% Configuration des titres
\titleformat{\chapter}[display]
{\normalfont\huge\bfseries\color{primaryblue}}
{\chaptertitlename\ \thechapter}{20pt}{\Huge}

\titleformat{\section}
{\normalfont\Large\bfseries\color{primaryblue}}
{\thesection}{1em}{}

\titleformat{\subsection}
{\normalfont\large\bfseries\color{secondaryblue}}
{\thesubsection}{1em}{}

% Configuration des liens
\hypersetup{
    colorlinks=true,
    linkcolor=primaryblue,
    filecolor=primaryblue,
    urlcolor=primaryblue,
    citecolor=primaryblue
}

% Commandes personnalisées
\newcommand{\code}[1]{\texttt{\color{primaryblue}#1}}
\newcommand{\file}[1]{\texttt{\color{secondaryblue}#1}}
\newcommand{\service}[1]{\textbf{\color{primaryblue}#1}}

\begin{document}

% Page de titre
\begin{titlepage}
    \centering
    \vspace*{2cm}
    
    {\Huge\bfseries\color{primaryblue} RAPPORT TECHNIQUE}\\[0.5cm]
    {\LARGE\color{secondaryblue} Système de Monitoring}\\[0.3cm]
    {\Large\color{darkgray} Prometheus, Grafana \& AlertManager}\\[2cm]
    
    \begin{tikzpicture}[scale=0.8]
        % Diagramme d'architecture simplifié
        \node[draw, rectangle, fill=primaryblue!20, minimum width=3cm, minimum height=1cm] at (0,0) {\textbf{Prometheus}};
        \node[draw, rectangle, fill=secondaryblue!20, minimum width=3cm, minimum height=1cm] at (5,0) {\textbf{Grafana}};
        \node[draw, rectangle, fill=primaryblue!20, minimum width=3cm, minimum height=1cm] at (2.5,-2) {\textbf{AlertManager}};
        
        \draw[->, thick] (1.5,0) -- (3.5,0);
        \draw[->, thick] (1.25,-0.5) -- (1.75,-1.5);
        \draw[->, thick] (3.75,-0.5) -- (3.25,-1.5);
    \end{tikzpicture}
    
    \vfill
    
    {\Large\textbf{Auteur:} Oussama Bartil}\\[0.3cm]
    {\Large\textbf{Date:} \today}\\[0.3cm]
    {\Large\textbf{Version:} 1.0}\\[2cm]
    
    \textcolor{darkgray}{\rule{\linewidth}{0.5mm}}\\[0.3cm]
    {\large\textcolor{darkgray}{Documentation technique complète du système de monitoring}}
\end{titlepage}

% Table des matières
\tableofcontents
\newpage

\chapter{Introduction}

\section{Objectif du Rapport}

Ce rapport présente une documentation technique complète du système de monitoring mis en place utilisant la stack Prometheus, Grafana et AlertManager. Il détaille l'architecture, la configuration, les métriques collectées, et les procédures de maintenance.

\section{Contexte du Projet}

Le système de monitoring a été développé pour surveiller en temps réel les performances d'un environnement Windows avec des conteneurs Docker. Il permet de :

\begin{itemize}
    \item Collecter des métriques système (CPU, mémoire, disque, réseau)
    \item Surveiller les conteneurs Docker
    \item Générer des alertes automatiques
    \item Visualiser les données via des dashboards interactifs
    \item Envoyer des notifications par email via Mailtrap
\end{itemize}

\section{Technologies Utilisées}

\begin{table}[H]
\centering
\begin{tabular}{|l|l|l|l|}
\hline
\textbf{Composant} & \textbf{Version} & \textbf{Rôle} & \textbf{Port} \\
\hline
Prometheus & latest & Collecte et stockage des métriques & 9090 \\
\hline
Grafana & latest & Visualisation et dashboards & 3000 \\
\hline
AlertManager & latest & Gestion des alertes & 9093 \\
\hline
cAdvisor & v0.47.0 & Métriques des conteneurs & 8082 \\
\hline
Windows Exporter & - & Métriques système Windows & 9182 \\
\hline
\end{tabular}
\caption{Technologies et composants du système}
\label{tab:technologies}
\end{table}

\chapter{Architecture du Système}

\section{Vue d'Ensemble}

Le système de monitoring suit une architecture modulaire basée sur des conteneurs Docker. Chaque composant a un rôle spécifique dans la chaîne de collecte, traitement et visualisation des données.

\begin{figure}[H]
\centering
\begin{tikzpicture}[scale=0.9]
    % Système Windows
    \node[draw, rectangle, fill=lightgray, minimum width=4cm, minimum height=2cm] at (0,0) {
        \begin{tabular}{c}
            \textbf{Système Windows} \\
            CPU, RAM, Disque \\
            Processus, Services
        \end{tabular}
    };
    
    % Windows Exporter
    \node[draw, rectangle, fill=yellow!30, minimum width=2.5cm, minimum height=1cm] at (0,-3) {\textbf{Windows Exporter}};
    
    % Conteneurs Docker
    \node[draw, rectangle, fill=blue!20, minimum width=3cm, minimum height=1.5cm] at (6,0) {
        \begin{tabular}{c}
            \textbf{Conteneurs Docker} \\
            Prometheus, Grafana \\
            AlertManager, cAdvisor
        \end{tabular}
    };
    
    % cAdvisor
    \node[draw, rectangle, fill=green!30, minimum width=2cm, minimum height=1cm] at (6,-3) {\textbf{cAdvisor}};
    
    % Prometheus
    \node[draw, rectangle, fill=red!30, minimum width=2.5cm, minimum height=1cm] at (3,-6) {\textbf{Prometheus}};
    
    % Grafana
    \node[draw, rectangle, fill=orange!30, minimum width=2cm, minimum height=1cm] at (0,-9) {\textbf{Grafana}};
    
    % AlertManager
    \node[draw, rectangle, fill=purple!30, minimum width=2.5cm, minimum height=1cm] at (6,-9) {\textbf{AlertManager}};
    
    % Mailtrap
    \node[draw, rectangle, fill=pink!30, minimum width=2cm, minimum height=1cm] at (9,-9) {\textbf{Mailtrap}};
    
    % Flèches
    \draw[->, thick] (0,-1) -- (0,-2);
    \draw[->, thick] (6,-1.5) -- (6,-2);
    \draw[->, thick] (0,-4) -- (2,-5.5);
    \draw[->, thick] (6,-4) -- (4,-5.5);
    \draw[->, thick] (2.5,-7) -- (1,-8);
    \draw[->, thick] (4.5,-7) -- (6,-8);
    \draw[->, thick] (6.5,-9) -- (8.5,-9);
    
    % Labels
    \node[color=darkgray] at (-1.5,-2.5) {\small Métriques};
    \node[color=darkgray] at (7.5,-2.5) {\small Métriques};
    \node[color=darkgray] at (1,-4.5) {\small Scraping};
    \node[color=darkgray] at (5,-4.5) {\small Scraping};
    \node[color=darkgray] at (1.5,-7.5) {\small Requêtes};
    \node[color=darkgray] at (5,-7.5) {\small Alertes};
    \node[color=darkgray] at (7.5,-8.5) {\small Email};
\end{tikzpicture}
\caption{Architecture générale du système de monitoring}
\label{fig:architecture}
\end{figure}

\section{Flux de Données}

Le flux de données suit le schéma suivant :

\begin{enumerate}
    \item \textbf{Collecte} : Les exporters (Windows Exporter, cAdvisor) collectent les métriques
    \item \textbf{Scraping} : Prometheus récupère les métriques via HTTP
    \item \textbf{Stockage} : Les données sont stockées dans la base de données time-series de Prometheus
    \item \textbf{Évaluation} : Les règles d'alertes sont évaluées périodiquement
    \item \textbf{Visualisation} : Grafana interroge Prometheus pour afficher les dashboards
    \item \textbf{Notification} : AlertManager envoie les alertes par email via Mailtrap
\end{enumerate}

\chapter{Configuration Détaillée}

\section{Configuration Prometheus}

Prometheus est configuré via le fichier \file{prometheus/prometheus.yml}. Cette configuration définit les cibles de scraping, les intervalles de collecte et les règles d'alertes.

\subsection{Configuration Globale}

La configuration globale définit les paramètres par défaut :
\begin{itemize}
    \item Intervalle de collecte : 15 secondes
    \item Intervalle d'évaluation des règles : 15 secondes
    \item Labels externes pour identifier le système
\end{itemize}

\subsection{Cibles de Scraping}

Le système collecte des métriques depuis plusieurs sources :

\begin{table}[H]
\centering
\begin{tabular}{|l|l|l|l|}
\hline
\textbf{Job} & \textbf{Cible} & \textbf{Intervalle} & \textbf{Métriques} \\
\hline
prometheus & localhost:9090 & 5s & Métriques internes \\
\hline
cadvisor & cadvisor:8080 & 5s & Conteneurs Docker \\
\hline
windows-exporter & host.docker.internal:9182 & 10s & Système Windows \\
\hline
grafana & grafana:3000 & 5s & Interface Grafana \\
\hline
\end{tabular}
\caption{Configuration des cibles de scraping}
\label{tab:scraping-targets}
\end{table}

\section{Configuration AlertManager}

AlertManager gère les alertes générées par Prometheus et les route vers les destinataires appropriés.

\subsection{Configuration SMTP}

AlertManager est configuré pour utiliser Mailtrap comme serveur SMTP :
\begin{itemize}
    \item Serveur : sandbox.smtp.mailtrap.io:2525
    \item Authentification avec nom d'utilisateur et mot de passe
    \item Adresse d'expéditeur : alertmanager@monitoring.local
\end{itemize}

\subsection{Routage des Alertes}

Les alertes sont routées selon leur type et leur sévérité :

\begin{itemize}
    \item \textbf{Alertes CPU} : Notification immédiate à \code{oussamabartil.04@gmail.com}
    \item \textbf{Alertes critiques} : Notification immédiate
    \item \textbf{Alertes warning} : Notification groupée toutes les 30 minutes
\end{itemize}

\chapter{Métriques et Alertes}

\section{Métriques Système}

Le système collecte diverses métriques pour surveiller les performances :

\subsection{Métriques CPU}

\begin{table}[H]
\centering
\begin{tabular}{|l|l|l|}
\hline
\textbf{Métrique} & \textbf{Description} & \textbf{Seuil d'Alerte} \\
\hline
node\_cpu\_seconds\_total & Temps CPU par mode & > 30\% pendant 2 min \\
\hline
windows\_cpu\_time\_total & Temps CPU Windows & > 30\% pendant 2 min \\
\hline
\end{tabular}
\caption{Métriques CPU surveillées}
\label{tab:cpu-metrics}
\end{table}

\subsection{Métriques Mémoire}

\begin{table}[H]
\centering
\begin{tabular}{|l|l|l|}
\hline
\textbf{Métrique} & \textbf{Description} & \textbf{Seuil d'Alerte} \\
\hline
node\_memory\_MemTotal\_bytes & Mémoire totale & - \\
\hline
node\_memory\_MemAvailable\_bytes & Mémoire disponible & < 15\% pendant 5 min \\
\hline
\end{tabular}
\caption{Métriques mémoire surveillées}
\label{tab:memory-metrics}
\end{table}

\section{Règles d'Alertes}

Les règles d'alertes sont définies dans \file{prometheus/rules/alerts.yml} et organisées en groupes :

\subsection{Alertes Système}

Les principales alertes système incluent :
\begin{itemize}
    \item \textbf{HighCPUUsage} : CPU > 30\% pendant 2 minutes
    \item \textbf{HighMemoryUsage} : Mémoire > 85\% pendant 5 minutes
    \item \textbf{LowDiskSpace} : Disque > 90\% pendant 5 minutes
    \item \textbf{ServiceDown} : Service indisponible > 1 minute
\end{itemize}

\subsection{Alertes Docker}

Le système surveille également les conteneurs Docker :

\begin{itemize}
    \item \textbf{ContainerDown} : Conteneur arrêté pendant plus d'1 minute
    \item \textbf{ContainerHighCPU} : CPU conteneur > 80\% pendant 5 minutes
    \item \textbf{ContainerHighMemory} : Mémoire conteneur > 90\% pendant 5 minutes
\end{itemize}

\chapter{Dashboards Grafana}

\section{Configuration des Sources de Données}

Grafana est configuré automatiquement avec Prometheus comme source de données principale via le provisioning.

\section{Dashboards Disponibles}

Le système inclut plusieurs dashboards pré-configurés :

\subsection{Dashboard Système}

\begin{itemize}
    \item Vue d'ensemble des performances système
    \item Graphiques CPU, mémoire, disque, réseau
    \item Métriques en temps réel et historiques
\end{itemize}

\subsection{Dashboard Docker}

\begin{itemize}
    \item État des conteneurs
    \item Utilisation des ressources par conteneur
    \item Métriques de performance des services
\end{itemize}

\chapter{Tests et Validation}

\section{Tests Automatisés}

Le système inclut plusieurs scripts de test pour valider le fonctionnement :

\begin{table}[H]
\centering
\begin{tabular}{|l|l|l|}
\hline
\textbf{Script} & \textbf{Fonction} & \textbf{Durée} \\
\hline
test-complete-monitoring.ps1 & Test complet du système & 2-3 min \\
\hline
test-mailtrap.ps1 & Test des notifications email & 1 min \\
\hline
cpu\_stress\_test.ps1 & Test de charge CPU & 10 min \\
\hline
health-check.ps1 & Vérification de santé & 30 sec \\
\hline
\end{tabular}
\caption{Scripts de test disponibles}
\label{tab:test-scripts}
\end{table}

\section{Procédures de Test}

\subsection{Test de Base}

Pour vérifier le fonctionnement de base du système, utilisez les commandes PowerShell appropriées pour vérifier l'état des services et tester les APIs.

\subsection{Test des Alertes}

Pour tester le système d'alertes, lancez le test de charge CPU et envoyez des alertes de test via les scripts fournis.

\chapter{Maintenance et Sauvegarde}

\section{Procédures de Sauvegarde}

Le système inclut un script de sauvegarde automatisé qui sauvegarde :

\begin{itemize}
    \item Configurations Prometheus, Grafana, AlertManager
    \item Données des bases de données
    \item Scripts et fichiers de configuration
    \item Dashboards personnalisés
\end{itemize}

\section{Maintenance Préventive}

\subsection{Vérifications Quotidiennes}

\begin{itemize}
    \item État des conteneurs Docker
    \item Espace disque disponible
    \item Logs d'erreur
    \item Performance des requêtes
\end{itemize}

\subsection{Vérifications Hebdomadaires}

\begin{itemize}
    \item Mise à jour des images Docker
    \item Nettoyage des données anciennes
    \item Test complet du système
    \item Vérification des sauvegardes
\end{itemize}

\chapter{Conclusion}

\section{Résumé du Système}

Le système de monitoring mis en place offre une solution complète et robuste pour la surveillance d'un environnement Windows avec conteneurs Docker. Il combine :

\begin{itemize}
    \item \textbf{Collecte automatisée} des métriques système et applicatives
    \item \textbf{Visualisation intuitive} via des dashboards Grafana
    \item \textbf{Alertes proactives} avec notifications par email
    \item \textbf{Architecture modulaire} facilement extensible
    \item \textbf{Procédures de maintenance} automatisées
\end{itemize}

\section{Performances et Métriques}

Le système démontre d'excellentes performances :

\begin{table}[H]
\centering
\begin{tabular}{|l|l|}
\hline
\textbf{Métrique} & \textbf{Valeur} \\
\hline
Temps de réponse moyen & < 100ms \\
\hline
Rétention des données & 200 heures \\
\hline
Fréquence de collecte & 15 secondes \\
\hline
Disponibilité & > 99.9\% \\
\hline
Temps de détection d'alerte & < 2 minutes \\
\hline
\end{tabular}
\caption{Performances du système}
\label{tab:performances}
\end{table}

\section{Évolutions Futures}

Plusieurs améliorations peuvent être envisagées :

\begin{itemize}
    \item \textbf{Monitoring applicatif} : Ajout de métriques métier
    \item \textbf{Intégration cloud} : Surveillance des services cloud
    \item \textbf{Machine Learning} : Détection d'anomalies automatique
    \item \textbf{Haute disponibilité} : Clustering Prometheus
    \item \textbf{Sécurité renforcée} : Authentification et chiffrement
\end{itemize}

\section{Recommandations}

Pour optimiser l'utilisation du système :

\begin{enumerate}
    \item \textbf{Formation} : Former les équipes à l'utilisation de Grafana
    \item \textbf{Documentation} : Maintenir la documentation à jour
    \item \textbf{Monitoring du monitoring} : Surveiller les performances du système lui-même
    \item \textbf{Tests réguliers} : Exécuter les tests automatisés hebdomadairement
    \item \textbf{Sauvegardes} : Vérifier régulièrement les procédures de sauvegarde
\end{enumerate}

\end{document}
